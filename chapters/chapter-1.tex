\section{Introducción}

Hola, este es el primer vistazo a un template de \LaTeX{} en desarrollo. Entre sus características destacables encontramos su dualidad de uso, ya sea para artículos o reportes\footnote{El uso de reporte puede ser muy adecuado para un formato de libro gracias a su jerarquía de \textit{Capítulos}} de \LaTeX{}, la creación automática de una portada simple con logo y una tabla de contenidos.

La cantidad de paquetes es muy reducida y los tiempos de compilación son suficientemente rápidos. Más guías de uso se pueden encontrar en el código de este proyecto. Futuras actualizaciones incluirán un manual de uso más detallado. 

Más información de uso en el respectivo repositorio de Github \cite{opc:autor}.

\begin{figure}[h]
    \centering
    \includegraphics[width=0.15\textwidth]{src/img/logo/author.pdf}
    \caption{Autor}
    \label{fig:authorvector}
\end{figure}

\newpage

\section{Matemáticas}

Esta plantilla posee amplio soporte para todo lo que respecta a Matemáticas y código fuente de diversos lenguajes. Por el apartado de Matemáticas es posible realizar bloques automáticos de Teoremas, Corolarios, Conjeturas, Proposiciones, Lemas, Preguntas, Afirmaciones, Propiedades, Definiciones, Condiciones, Notaciones, Soluciones, Observaciones, Ejemplos y Ejercicios. Estos bloques cuentan con un sistema de enumeración automático, lo que será de gran ayuda para textos largos. Ejemplo:



\begin{thm}[Algebra de derivadas]
    Si $f$ y $g$ son diferenciables en $x_0$ y $\alpha \in \mathbb{R}$, entonces $f \pm g$, $\alpha g$, $fg$ y $f/g$ con $g(x_0) \neq 0$ son también diferenciables y además:

    \begin{enumerate}
        \item $(f \pm g)^\prime = f^\prime \pm g^\prime$
        \item $(\alpha f)^\prime = \alpha f^\prime$
        \item $(fg)^\prime = f^\prime g + fg^\prime$
        \item $(f/g)^\prime = \dfrac{f^\prime g - fg^\prime}{g^2}$
    \end{enumerate}
\end{thm}

\begin{proof}
    \begin{enumerate}
        \item 
        \begin{align*}
            (f \pm g)^\prime & = \lim_{h \to 0} \dfrac{(f \pm g)(x+h)-(f \pm g)(x)}{h} \\
            & = \lim_{h \to 0} \dfrac{f(x+h)-f(x)}{h} \pm \lim_{h \to 0} \dfrac{g(x+h)-g(x)}{h} \\
            & = f^\prime \pm g^\prime \\
            & = (f \pm g)^\prime 
        \end{align*}
        \item Propuesta
        \item Propuesta
        \item Propuesta
    \end{enumerate}
\end{proof}

\newpage

\begin{cor}
    $\left(\dfrac{1}{f}\right)^\prime = -\dfrac{f^\prime}{f^2}$
\end{cor}

\begin{exs}\leavevmode
    \begin{enumerate}
        \item $\dfrac{d}{dx}(\tan{x}) = \sec^2{x}$
        \item $\dfrac{d}{dx}(\sec{x}) = \sec{x}\tan{x}$
        \item $\dots$
    \end{enumerate}
\end{exs}

\section{Ejemplos de código}

Si bien el código en \texttt{PDF} no suele ser muy agradable de manejar, es una característica que existe y tiene coloreado gracias a \href{https://pypi.org/project/Pygments/}{\texttt{Pygments}} (\mintinline{text}|pip install Pygments|). y el paquete \href{https://ctan.org/pkg/minted}{\texttt{minted}} de \LaTeX{}. Veamos un ejemplo de código en \texttt{C}:

\begin{listing}[H]
    \centering
    \inputminted[frame=single]{c}{src/code/helloworld.c}
    \label{lst:the-code}
    \caption{Código de ``Hello World'' en C}
\end{listing}

El código anterior es proporcionado por el archivo \texttt{helloworld.c}, mientras que el siguiente, escrito en \texttt{Python} simplemente se escribe en el entorno básico del paquete:

\begin{listing}[H]
\begin{minted}{python}
n1 = float(input("Ingrese número uno: "))
n2 = float(input("Ingrese numero dos: "))
suma = n1 + n2

print("La suma es: ", suma)
\end{minted}
\label{}
\caption{Ejemplo de suma de dos enteros en C}
\end{listing}

\newpage

\section{Algoritmos}

Similar al código fuente.

\begin{algorithm}[H]
\caption{Un algoritmo con descripción}\label{alg:cap}
\begin{algorithmic}
\Require $n \geq 0$
\Ensure $y = x^n$
\State $y \gets 1$
\State $X \gets x$
\State $N \gets n$
\While{$N \neq 0$}
\If{$N$ is even}
    \State $X \gets X \times X$
    \State $N \gets \frac{N}{2}$  \Comment{Esto es un comentario}
\ElsIf{$N$ is odd}
    \State $y \gets y \times X$
    \State $N \gets N - 1$
\EndIf
\EndWhile
\end{algorithmic}
\end{algorithm}
