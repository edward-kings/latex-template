\section{Introducción}

\noindent
Hola, este es el primer vistazo a un template de \LaTeX{} en desarrollo. Entre sus características destacables encontramos su dualidad de uso, ya sea para artículos o reportes\footnote{El uso de reporte puede ser muy adecuado para un formato de libro gracias a su jerarquía de \textit{Capítulos}} de \LaTeX{}, la creación automática de una portada simple con logo y una tabla de contenidos.

La cantidad de paquetes es muy reducida y los tiempos de compilación son extremadamente rápidos. Más guías de uso se pueden encontrar en el código de este proyecto. Futuras actualizaciones incluirán un manual de uso más detallado. 

Más información de uso en el respectivo repositorio de Github \cite{opc:EduardoReyes}.

\begin{figure}[h]
    \centering
    \includegraphics[width=0.25\textwidth]{src/img/logo/author.pdf}
    \caption{Eduardo Reyes}
    \label{fig:authorvector}
\end{figure}

\newpage

\section{Matemáticas}

\noindent
Esta plantilla posee amplio soporte para todo lo que respecta a Matemáticas y código fuente de diversos lenguajes. Por el apartado de Matemáticas es posible realizar bloques automáticos de Teoremas, Corolarios, Conjeturas, Proposiciones, Lemas, Preguntas, Afirmaciones, Propiedades, Definiciones, Condiciones, Notaciones, Soluciones, Observaciones, Ejemplos y Ejercicios. Estos bloques cuentan con un sistema de enumeración automática por capítulo, lo que será de gran ayuda para textos largos.

Veamos su uso:

\begin{thm}[Algebra de derivadas]
    Si $f$ y $g$ son diferenciables en $x_0$ y $\alpha \in \mathbb{R}$, entonces $f \pm g$, $\alpha g$, $fg$ y $f/g$ con $g(x_0) \neq 0$ son también diferenciables y además:

    \begin{enumerate}
        \item $(f \pm g)^\prime = f^\prime \pm g^\prime$
        \item $(\alpha f)^\prime = \alpha f^\prime$
        \item $(fg)^\prime = f^\prime g + fg^\prime$
        \item $(f/g)^\prime = \dfrac{f^\prime g - fg^\prime}{g^2}$
    \end{enumerate}
\end{thm}

\begin{proof}
    \begin{enumerate}
        \item 
        \begin{align*}
            (f \pm g)^\prime & = \lim_{h \to 0} \dfrac{(f \pm g)(x+h)-(f \pm g)(x)}{h} \\
            & = \lim_{h \to 0} \dfrac{f(x+h)-f(x)}{h} \pm \lim_{h \to 0} \dfrac{g(x+h)-g(x)}{h} \\
            & = f^\prime \pm g^\prime \\
            & = (f \pm g)^\prime 
        \end{align*}
        \item Propuesta
        \item Propuesta
        \item Propuesta
    \end{enumerate}
\end{proof}

\begin{cor}
    $\left(\dfrac{1}{f}\right)^\prime = -\dfrac{f^\prime}{f^2}$
\end{cor}

\begin{exs}\leavevmode
    \begin{enumerate}
        \item $\dfrac{d}{dx}(\tan{x}) = \sec^2{x}$
        \item $\dfrac{d}{dx}(\sec{x}) = \sec{x}\tan{x}$
        \item $\dots$
    \end{enumerate}
\end{exs}

\newpage

\section{Ejemplos de código}

\noindent
Si bien el código en \texttt{PDF} no suele ser muy agradable de manejar, es una característica que existe y tiene coloreado gracias a \texttt{Pygments} (\mintinline{text}|pip install Pygments|). y el paquete \texttt{minted} de \LaTeX{}. Veamos un ejemplo de código en \texttt{C}: \\

\begin{codecaption}
    \begin{mintedcode}{c}
        #include <stdio.h>
        int main() {    

            int number1, number2, sum;
            
            printf("Enter two integers: ");
            scanf("%d %d", &number1, &number2);

            // calculating sum
            sum = number1 + number2;      
            
            printf("%d + %d = %d", number1, number2, sum);
            return 0;
        }
    \end{mintedcode}
\caption{Ejemplo de suma de dos enteros en C}
\end{codecaption}

Ahora un ejemplo en \texttt{SQL}: \\

\begin{codecaption}
    \begin{mintedcode}{sql}
        SELECT nombre AS planeta
        FROM Planeta
        UNION ALL
        SELECT planeta
        FROM Satélite  
    \end{mintedcode}
\caption{Ejemplo de consulta en SQL}
\end{codecaption}

Además es posible insertar código desde archivos gracias a \texttt{inputminted}: \\

\begin{codecaption}
    \begin{inputcode}
        \inputminted{c}{src/code/helloworld.c}
    \end{inputcode}
\caption{Ejemplo de Hello World! en C}
\end{codecaption}